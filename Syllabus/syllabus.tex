% !TeX program = lualatex
\documentclass[11pt,letterpaper]{article}

\usepackage{fontspec}
\usepackage{tabularx}
\usepackage{booktabs}
\usepackage{bucolors}
\usepackage[colorlinks=true,allcolors=black,urlcolor=bugold]{hyperref}
\usepackage{mathtools}
\usepackage{mdframed}
\usepackage{titling}
\usepackage{fancyhdr}
\usepackage{lastpage}
\usepackage[english]{babel}
\usepackage[sf,bf]{titlesec}
\usepackage[inline]{enumitem}
\usepackage[margin=1in,letterpaper]{geometry}
\usepackage{multicol}
\usepackage{soul}
\usepackage{nameref}

\setmainfont{TeX Gyre Termes}[Ligatures=TeX]
\setsansfont{TeX Gyre Heros}[Ligatures=TeX]
\setmonofont{TeX Gyre Cursor}

\title{How and Why Things Work:\\ A Course for the Curious}
\author{Dr.\ Daniel A.\ McCurry}
\date{Fall 2023}

\newcommand{\classnum}{FYS100}

\pagestyle{fancy}
\lhead{}
\chead{}
\rhead{}
\lfoot{\footnotesize\sffamily McCurry --- \classnum\ --- FA23}
\cfoot{}
\rfoot{\footnotesize\sffamily\thepage~of~\pageref{LastPage}}
\renewcommand{\headrulewidth}{0pt}
\renewcommand{\footrulewidth}{0.4pt}

\pretitle{\noindent\color{bumaroon}
	\sffamily\bfseries\Large
	\classnum \newline
	\LARGE\expandafter\MakeUppercase\expandafter}
\posttitle{\par\medskip}
\preauthor{\noindent\sffamily}
\postauthor{ --- }
\predate{\sffamily}
\postdate{}

\setlength{\droptitle}{-4em}

\setcounter{secnumdepth}{0}
\setlist[description]{font=\sffamily\bfseries\small}
\urlstyle{same}

\begin{document}

\maketitle
\thispagestyle{fancy}

\noindent
\begin{tabularx}{\linewidth} {@{\qquad}>{\bfseries\sffamily}r
	*{2}{>{\raggedright\arraybackslash}X}}
	\toprule
	Day and Time: & Section 10 & Section 11 \\
		      & 9:00 am -- 9:50 am & 11:00 am -- 11:50 am \\ 
		 & MWF @ HSC  & MWF @ HSC  \\ \\
	Instructor: & Dr.\ Daniel A. McCurry &
	\href{mailto:dmccurry@commonwealthu.edu}{\nolinkurl{dmccurry@commonwealthu.edu}}
	\\
		    & 	Associate Professor of Chemistry & (570) 389-5320 \\
		    & 	HSC 240 &
		    \href{https://bloomu.starfishsolutions.com/starfish-ops/dl/instructor/serviceCatalog.html?bookmark=connection/20001}{CU
			    Succeed
		     	Profile} \\ \\
	Office Hours: & 
		\begin{tabular}[t] {@{}lr@{\,--\,}l}
			Mon./Tue.  & 3:00 & 5:00\,p.m. \\
			Thur. & 4:00 & 5:00\,p.m. \\ 
		\end{tabular}
		      & Need another time? \newline
\href{https://outlook.office.com/bookwithme/user/38fcfec5771549768d6c1a6f66912778@commonwealthu.edu/meetingtype/scMRb4P_HUKzXVNHurMbRA2?anonymous}{Schedule
			an Appointment}
			 \\ \\
%		      &   In-Person or via Zoom Meeting ID
%                          ``\href{https://bloomu.zoom.us/my/dmccurry}{dmccurry}''\\
%                      &    \href{https://bloomu.starfishsolutions.com/starfish-ops/dl/instructor/serviceCatalog.html?bookmark=connection/20001/schedule}{Schedule
%		      an Appointment} (not required) \\ \\
%		Textbook: %& Harvey, D. \textit{Analytical Chemistry 2.1} (\href{https://chem.libretexts.org/@go/page/122341}{Open Access}) \\
%			   & \multicolumn{2}{l}{Harris, D.C. \textit{Quantitative Chemical Analysis},
%	10\textsuperscript{th} Ed.}  \\ & \multicolumn{2}{l}{(older editions OK, Inclusive
%Access opt out by \textbf{8/28})}
%	      \\ \\
	Materials: & \multicolumn{2}{l}{Laptop} \\
	\bottomrule
\end{tabularx}

\section{Course Description}
Designed for first-year students, this course cultivates scholarly and academic
succes, promotes engagement with the univerity community, fosters personal
development and wellness, and promotes understanding of diversity and social
responsibility. Individual sections of the course are devoted to specific themes
selected by instructors and of interest to students as they begin the
intentional process of degree and career planning. The course is required for
all first-year students.

Have you ever observed something in everyday life and wondered, ``Why/How did
that happen?'' This is a course for those not just interested in observing, but
knowing the underlying scientific reasoning behind everyday events. Most
phenomena can be explained by investigating the interaction of molecules and
basic chemical principles. Some results will be urprising, but when basic
cientific principle are applied, the results can be explained and understood.

\subsubsection{Potential questions we answer may include the following:}
\begin{itemize}
	\item How does soap know what is dirt?
	\item How do microwaves work?
	\item Why i there wood in my shredded cheese?
	\item Why doesn't your car start in cold weather?
	\item Why do waves roll parallel to the shore?
	\item How can you tell temperature by listening to a cricket?
	\item How did that latest TikTok life hack actually work?
	\item What exactly is in a Red Bull?
	\item Exactly how much sleep do I need to function?
	\item Why is there so much saturated fat in Ramen noodles?
	\item Any other topics you are curious about.
\end{itemize}

These questions will be answered using the scientific method and classroom
demonstrations/lab as applicable.

This specific theme will be interwoven (where appropriate) with the following
Firt Year Seminar Course Goals and Content. The course will also use guest
speakers/field trips to multiple areas on campus, which may include Alumni and
Professional Engagement, The Learning Center, WALES, Student Success, Counseling
services, financil aid, Health and Wellness, and Diversity Equity, and
Inclusion. One or more trips to the library is also planned.

\subsection{Learning Objectives}
By the end of this course, you are expected to be able to:
\begin{itemize}[noitemsep]
	\item Determine the extent of information needed.
	\item Assess the needed information using current technologies.
	\item Evaluate information and its sources critically.
	\item Use information effectively to present a literature review.
	\item Access and use information ethically and legally.
\end{itemize}

\subsection{Course Community and Communication}
In a remote instruction setting, it can become incredibly difficult to build
personal connections with your peers. To encourage some communication, Dr.\
McCurry will be using the Discussion Board on BOLT extensively. If you have a
question with an answer that could benefit the class as a whole, please post
your question to the Discussion Board, rather than sending an individual email
to Dr.\ McCurry. I would also be happy to help organize study groups or post
links to class chat groups (Google Meet, Discord, Matrix, etc.) if you are
interested. Please let me know if you would like to chat with other classmates
and I will put you in touch.

All course announcements will be delivered on the BOLT course page. It is your
responsibility to regularly check BOLT. Dr.\ McCurry will \emph{not} send out
email announcements, so BOLT is the \emph{only} location you will need to check
for all important course information. If you would like push notifications on
your phone when announcements are posted, please download the
\href{https://documentation.brightspace.com/EN/brightspace/requirements/all/pulse.htm}{Brightspace
Pulse} app.

\section{Evaluations and Grading}
Your overall course grade will be calculated as follows:
\begin{multicols}{2}
	\subsection{Point Distribution}
	\begin{center}
		\begin{tabular}{l r<{\%}}
			In-Class Discussion & 20 \\
			Written Assignments & 35 \\
			Final Project & 45 \\ \midrule
			& 100
		\end{tabular}
	\end{center}
	\subsection{Grade Assignment}
	\begin{center}
		\begin{tabular} {r@{~--~}l l @{\hspace{0.5in}}r@{~--~}l l}
			\multicolumn{3}{c}{} & 76 & 77 &  C+ \\
		 	90 & 100 & A         & 67 & 75 &  C  \\
		 	88 & 89  & A-        & 65 & 66 &  C- \\
		 	86 & 87  & B+        & 63 & 64 &  D+ \\
			80 & 85  & B         & 55 & 62 &  D  \\
		 	78 & 79  & B-        &  0 & 54 &  F  \\
	\end{tabular}
\end{center}
\end{multicols}

\subsection{Graded Items}
\begin{description}
	\item[In-Class Discussions:]
		Most weeks we will spend some class time discussing at least one
		specific piece of scientific literature, whether it be a journal
		article, a news article dealing with a scientific topic, or some
		other document dealing with an aspect of critical reading,
		writing, and thinking. Your grade on these assignments will not
		be based on how much of the given article you comprehend but
		instead on how actively you engage in classroom discussions
		related to the material. As such, it is important that you read
		these articles prior to coming to class and come prepared to
		discuss them. I would also recommend taking notes as you read
		these articles; this will aid your participation in the class
		discussion.
		\begin{itemize}
			\item For each class period you will be assigned a
				participation score of 0--3 points based upon
				your level of participation. Points are
				generally assigned as follows: Zero points for
				an unexcused absence; one point for attending
				class with no participation; two points for
				participation that was initiated by instructor
				(i.e., being called upon); three points for
				voluntary participation.
			\item At the end of the semester, I will sum your
				participation points, divide by 42 (14 wks.
				$\times$ 3
				pts. each) and then multiply by 100 to determine
				your percentage score.
			\item If I plan for us to discuss specific articles
				during a class period, I will post the material
				on BOLT at least a week in advance.
		\end{itemize}
	\item[Written Assignments:] At various points during the semester, you
		will be required to submit short written assignments and/or
		reflections. These assignments will be aimed at developing your
		critical reading and writing skills by analyzing various aspects
		of the chemical literature or readings related to topics we have
		discussed in class. These assignments will either be made in
		class or announced on BOLT. In either case, you will have from
		one to two weeks to complete these assignments. Further details
		on each assignment will be provided as relevant.
		\begin{itemize}
%			\item As part of the course, \textbf{you are required to
%				attend all chemistry seminars}. The seminars are
%				typically scheduled on Friday afternoons at 4 pm
%				and will be announced in advance. You must have
%				a valid written excuse if you have a conflict
%				(\emph{verified in writing in advance}),
%				whereupon an alternative writing assignment will
%				be given.  Following the seminar, you will be
%				required to write a report of \emph{500-650}
%				words summarizing the topic presented and
%				offering some critique of the presentation
%				itself (i.e. graphs/text too small, unclear
%				speech, etc.).
			\item All written assignments will be will be worth 30
				points. The grading of these assignments will
				follow the guidelines given below:
				\begin{center}
					\begin{tabular}
						{>{\bfseries\sffamily}r<{\,pts.}
							l
						}
						10 & appropriate interpretation
						and summation of the assigned
						reading \\
						15 & critical analysis of or
						reflection on the assigned
						reading content \\
						5 & correct grammar,
						punctuation, syntax, etc.
					\end{tabular}
				\end{center}
			\item Assignments that do not meet the minimum length
				requirements (as stated when the assignment is
				given) will be penalized accordingly.
		\end{itemize}
	\item[Final Project:] In place of a final exam, you will be required to
		use a popular science article and at least one accompanying
		primary research article to complete a written assignment and
		class presentation. Full details regarding format, grading, and
		relevant due dates for the project will be posted to BOLT.
		\begin{itemize}
			\item In addition, you will be required to submit a
				paper summarizing your findings. This paper will
				be graded as above, with two important changes:
				(1) this final paper will be worth 100 points,
				and (2) these points will count toward your
				final project grade.
			\item Full details regarding the format, grading, and
				relevant due dates for the final project are
				provided in a separate document.
		\end{itemize}
\end{description}

\subsection{BOLT Submissions}

All written work in this course is to be submitted in PDF format via
BOLT. Please make sure that your name also appears \textbf{at the top}
of each submitted assignment.

\subsection{Late Submissions}

Written work submitted after the due date and time will be penalized 5\%
for every hour it is late for the first six hours and 10\% for every
hour thereafter.


\subsection{Grading Errors}
It is \emph{your} responsibility to bring grading errors to my attention. You
have 72 hours from the return of the graded item to submit it for a re-grade.
\emph{The entire item will be re-graded}. If the grading error results from a
technology failure, the professor may administer a make-up question. You have
until 4:30~p.m.\ on the day of the final exam to bring clerical errors to my
attention.

\section{Important Dates}\label{importantdates}
Please review the BOLT calendar as it includes all due dates for assignments as
well as our lecture schedule. Some important dates to keep in mind:
\begin{center}
	\begin{tabular} {l l l}
		February  5 & Friday & Tech Module due at 5:00 p.m. \\
		April 19 & Monday & Presentations \\
		April 26 & Monday & Presentations \\
		May 3 & Monday & Presentations \\
		May 10 & Monday & Final paper due \\
	\end{tabular}
\end{center}

Please note, if you are unable to complete an assignment by
its scheduled time due to some unavoidable circumstance, you are obligated to
provide a documented valid excuse to your instructor. \emph{Failure to provide a
valid excuse on these terms will result in a score of zero for the assignment.}

\subsection{Attendance}

Attendance in this course is mandatory. For every unexcused absence you
accumulate, two percentage points will be deducted from your course
average. Of course, if you have a \textbf{documented} valid excuse, you
will not be penalized. Valid excuses for missed class periods are:


	\begin{center}
		\renewcommand\arraystretch{1.25}
\begin{tabularx}{\linewidth} {X X}
	\toprule
	\bfseries Valid Excuse & \bfseries Time Frame to Provide Documentation
	\\ \midrule
	Personal illness or accident, illness or accident of a dependent child,
	or death or critical illness of an immediate family member &
	48\,hr after the missed graded event \\
	Participation in a university-sponsored activity &
	48\,hr in advance of the missed graded event \\
	Military duty & As soon as possible \\
	Religious observance & The second Friday of the semester \\
	Others, as merited on a case-by-case basis & 48\,hr after the
	missed graded event \\
	\bottomrule
\end{tabularx}
\end{center}

Note that technical issues, including internet connection problems, are
\emph{not} generally considered valid excuses. This course requires a regularly
available high speed internet connection. It is highly recommended that you do
not wait until the last minute to start an exam or assignment to avoid potential
issues. If you are not able to secure a stable connection and laptop or desktop
computer, please visit \url{https://bloomu.edu/student-assistance} and ``raise
your hand''. You may be eligible for a loaned laptop or cellular access point.

Immediate technical issues, such as BOLT being inaccessible, can be submitted to
\url{https://helpdesk.bloomu.edu}. At the bottom of the form, you must include
my email address (dmccurry@bloomu.edu) in the ``Additional Email Notifications''
to alert me that you are taking steps to resolve the issue.

\section{Policies and Expectations}
In order to maintain an atmosphere that is conducive to learning, all members
of the class are expected to demonstrate common courtesy during the lecture,
including but not exclusively:
\begin{enumerate}
	\item \textbf{Be on time for class.} In the rare instances where
		arriving late is unavoidable, enter quietly. Make sure your
		microphone is muted.
	\item \textbf{Once in class, stay for the duration.} If you know you
		need to leave early, please make me aware of your plans before
		class begins.
	\item \textbf{Be courteous with others on Zoom.} Allow others to finish
		speaking before unmuting yourself. If you have a question or
		would like me to repeat something, just use the raise your hand
		button in the participants list. (Do feel free to stop me if
		it's clear I didn't notice!)
	\item \textbf{Do not participate in activities not pertinent to the
		class.} Please focus your attention to the Zoom meeting.
	\item \textbf{Academic dishonesty is not tolerated.} All work in this
		course must be your own; on group projects your individual
		contribution to the project must be evident. You will be
		assigned a grade of ``F'' for the course if found violating the
		Academic Integrity policy (PRP3512).  See me if you are unclear
		about what I consider to be dishonest.
\end{enumerate}



\section{Academic Dishonesty}
\emph{Academic dishonesty is not tolerated.} If you are unclear about what is
dishonest, please see
\href{https://intranet.bloomu.edu/policies_procedures}{PRP 3512 --- Academic
Integrity Policy} for clarification. If you are unsure about my specific
instructions, ask me.

\begin{mdframed}
	\centering\bfseries The minimum penalty for academic dishonesty is a
	course assignment of ``F'' for \emph{all} students involved.
\end{mdframed}

\section{Class Cancellation}
If class is canceled, please check BOLT for course announcements. An assignment
may be provided to make up for the lost time. 
Please let your instructor know as soon as possible if you lose power or
internet access.

%\section{Fire Alarms}
%In the event of a building evacuation, calmly and quickly leave the building via
%the nearest exit. In the event of an evacuation during laboratory, ensure all
%hot plates or other possible hazards are turned off and unplugged prior to
%leaving. Your instructor will point exits out the first time you are physically
%in lab.  Gather with your class on the quad lawn in front of Bakeless.  Do not
%re-enter the building for any reason. Do not gather around the exits. Do not
%enter a building that has an alarm sounding. There is one, and only one, drill
%the first week of the semester. There are \emph{no} random drills.


\section{Last Day to Withdraw}
The Registrar has set the last day of class as the withdrawal deadline.
Students are encouraged to review the latest withdrawal
(\href{https://intranet.bloomu.edu/policies_procedures}{PRP 3462 ---
Undergraduate Course Withdrawal}) and course repeat
(\href{https://intranet.bloomu.edu/policies_procedures}{PRP 3452 --- Course
Repeat}) policies of the University prior to that date as there are strict
limits on the number of repeats one can have. 

\begin{mdframed}
	\centering
	All policies, rules, and procedures (PRP) are available at
	\url{https://www.bloomu.edu/policies_procedures}
\end{mdframed}

\section{Accommodations}
Bloomsburg University provides reasonable accommodations to students who have
documented disabilities. If you have a documented disability that requires
academic accommodations and are not registered with 
\href{https://intranet.bloomu.edu/disabilities}{University Disability Services},
Office, please contact this office as soon as possible to establish your
eligibility. Please also provide documentation to Dr.\ McCurry as promptly as
possible.

\vfill

\begin{mdframed}
	\noindent
	The materials contained in this syllabus and the
	BOLT webpage for this course are intended only for those registered for
	the above course/semester. The materials cannot be used without the
	expressed written consent of Dr.\ McCurry. Any use of these materials
	outside the intended purpose of this course is considered Copyright
	Infringement and will be dealt with according the Academic Integrity
	Policy.  

	\noindent
	The source code for some material has been licensed under the Creative
	Commons Attribution-NonCommercial-ShareAlike 4.0 International (CC
	BY-NC-SA 4.0) license to provide you with an opportunity to view the
	original source materials and contribute to the content. If you would
	like to view the material and suggest improvements, please visit
	\url{https://git.damccurry.com/dan/CHEM281}. Ask Dr.\ McCurry for an
	account!
\end{mdframed}

\end{document}
