% !TeX program = lualatex
\documentclass[11pt,letterpaper]{article}

\usepackage{fontspec}
\usepackage{tabularx}
\usepackage{booktabs}
\usepackage{bucolors}
\usepackage[colorlinks=true,allcolors=black,urlcolor=bugold]{hyperref}
\usepackage{mathtools}
\usepackage{mdframed}
\usepackage{titling}
\usepackage{fancyhdr}
\usepackage{lastpage}
\usepackage[english]{babel}
\usepackage[sf,bf]{titlesec}
\usepackage[inline]{enumitem}
\usepackage[margin=1in,letterpaper]{geometry}
\usepackage{multicol}
\usepackage{soul}
\usepackage{nameref}

\setmainfont{TeX Gyre Termes}[Ligatures=TeX]
\setsansfont{TeX Gyre Heros}[Ligatures=TeX]
\setmonofont{TeX Gyre Cursor}

\title{How and Why Things Work:\\ A Course for the Curious}
\author{Dr.\ Daniel A.\ McCurry}
\date{Fall 2023}

\newcommand{\classnum}{FYS100}

\pagestyle{fancy}
\lhead{}
\chead{}
\rhead{}
\lfoot{\footnotesize\sffamily McCurry --- \classnum\ --- FA23}
\cfoot{}
\rfoot{\footnotesize\sffamily\thepage~of~\pageref{LastPage}}
\renewcommand{\headrulewidth}{0pt}
\renewcommand{\footrulewidth}{0.4pt}

\pretitle{\noindent\color{bumaroon}
	\sffamily\bfseries\Large
	\classnum \newline
	\LARGE\expandafter\MakeUppercase\expandafter}
\posttitle{\par\medskip}
\preauthor{\noindent\sffamily}
\postauthor{ --- }
\predate{\sffamily}
\postdate{}

\setlength{\droptitle}{-4em}

\setcounter{secnumdepth}{0}
\setlist[description]{font=\sffamily\bfseries\small}
\urlstyle{same}

\begin{document}

\maketitle
\thispagestyle{fancy}

\noindent
\begin{tabularx}{\linewidth} {@{\qquad}>{\bfseries\sffamily}r
	*{2}{>{\raggedright\arraybackslash}X}}
	\toprule
	Day and Time: & Section 10 & Section 11 \\
		      & 9:00 am -- 9:50 am & 11:00 am -- 11:50 am \\ 
		 & MWF @ HSC 122  & MWF @ HSC B36  \\ 
		 & \multicolumn{2}{l}{\em\small You must attend the section for which you
	are registered --- see
\href{https://studentssb-prod.ec.passhe.edu/StudentSelfService/ssb/studentProfile/?mepCode=003315}{Banner}} \\ \\
	Instructor: & Dr.\ Daniel A. McCurry &
	\href{mailto:dmccurry@commonwealthu.edu}{\nolinkurl{dmccurry@commonwealthu.edu}}
	\\
		    & 	Associate Professor of Chemistry & (570) 389-5320 \\
		    & 	HSC 240 &
		    \href{https://bloomu.starfishsolutions.com/starfish-ops/dl/instructor/serviceCatalog.html?bookmark=connection/20001}{CU
			    Succeed
		     	Profile} \\ \\
	Office Hours: & 
		\begin{tabular}[t] {@{}lr@{\,--\,}l}
			Mon./Tue.  & 3:00 & 5:00\,p.m. \\
			Thur. & 9:00 & 10:00\,a.m. \\ 
		\end{tabular}
		      & Need another time? \newline
      		      \href{https://outlook.office.com/bookwithme/user/38fcfec5771549768d6c1a6f66912778@commonwealthu.edu/meetingtype/3ML2Y3JdEkuBxgdc6YpfzQ2?anonymous}{Schedule an Appointment}
			 \\ \\
	Materials: & \multicolumn{2}{l}{Laptop (for some topics)} \\
	\bottomrule
\end{tabularx}

\section{Course Description}
Designed for first-year students, this course cultivates scholarly and academic
success, promotes engagement with the university community, fosters personal
development and wellness, and promotes understanding of diversity and social
responsibility. Individual sections of the course are devoted to specific themes
selected by instructors and of interest to students as they begin the
intentional process of degree and career planning. The course is required for
all first-year students.

\subsection{How and Why Things Work}
Have you ever observed something in everyday life and wondered, ``Why/How did
that happen?'' This is a course for those not just interested in observing, but
knowing the underlying scientific reasoning behind everyday events. Most
phenomena can be explained by investigating the interaction of molecules and
basic chemical principles. Some results will be surprising, but when basic
scientific principle are applied, the results can be explained and understood.

Potential questions we answer may include the following:
\begin{multicols}{2}
\begin{itemize}[nosep]
	\item How does soap know what is dirt?
	\item How do microwaves work?
	\item Why is there wood in my shredded cheese?
	\item Why doesn't your car start in cold weather?
	\item Why do waves roll parallel to the shore?
	\item How can you tell temperature by listening to a cricket?
	\item How did that latest TikTok life hack actually work?
	\item What exactly is in an energy drink?
	\item Exactly how much sleep do I need to function?
	\item Why is there so much saturated fat in ramen noodles?
	\item Any other topics you are curious about.
\end{itemize}
\end{multicols}
These questions will be answered using the scientific method and classroom
demonstrations/lab as applicable. This specific theme will be interwoven (where
appropriate) with the following First Year Seminar Course Goals and Content. The
course will also use guest speakers/field trips to multiple areas on campus,
which may include Alumni and Professional Engagement, The Learning Center,
WALES, Student Success, Counseling services, Financial Aid, Health and Wellness,
and Diversity Equity, and Inclusion. One or more trips to the library is also
planned.

\subsection{FYS Course Goals and Content}
In addition to themed content, the following areas of study will be included:
\begin{multicols}{2}
	\raggedright
	\begin{itemize}[nosep]
		\item Basics for College Success
		\item Time Management
		\item Thriving in College --- Resiliency
		\item Critical Thinking
		\item Learning About Learning --- Metacognition
		\item Study Skills
		\item Academic Success
		\item Citizenship and Engagement
		\item Diversity, Race, and Inclusion
		\item Careers and Major Pathways
		\item Degree Completion Planning
		\item Course Registration and Working with Advisors
		\item Financial Wellness
		\item Health and Wellness
		\item Exploration of Campus Life
	\end{itemize}
\end{multicols}

\subsection{Learning Objectives}
Your success in this course will be based on your ability to meet the below
learning objectives:
\begin{enumerate}
	\item Utilize University resources for the purpose of academic support,
		selection of program of study, and location of information on
		campus life.
	\item Explore connections and opportunities among peers, mentors, and
		faculty in developing first-year experience as a University
		student.
	\item Develop and apply skills and strategies for critical thinking,
		resiliency, and mental health.
	\item Build comprehension of practices of diversity, inclusion, and
		overall positive citizenship, leadership, and engagement, on
		campus and beyond.
\end{enumerate}

\subsection{Course Community and Communication}
\emph{Your active involvement in the course is important for your success!} As
such, you are expected to read any assigned readings prior to lecture.
Please use the Discussion Boards on Brightspace for questions so other students
can contribute to or read answers. All course announcements will be delivered on
the Brightspace course page. If you would like push
notifications on your phone when announcements are posted, please download the
\href{https://documentation.brightspace.com/EN/brightspace/requirements/all/pulse.htm}{Brightspace
Pulse} app.


\section{Evaluations and Grading}
This course will be operating on a 100\,\% scale according to the following tables.
An emphasis will be placed on meeting the objectives for the course, so you
should expect to do well if you satisfy the criteria for each assignment and put
learned success strategies into practice. \textbf{My goal is for every student
	to get an A in this course and, as a result, an A in all their other
courses!} Note that this statement does \emph{not} guarantee an A in this course
--- if you are not doing well, that is a \emph{strong} sign that changes need to
be made! Please come see me in office hours or make an appointment. It's always
better to tackle issues sooner rather than later.

\begin{multicols}{2}
	\subsection{Point Distribution}
	\begin{center}
		\begin{tabular}{l r<{\%}}
			Attendance & 5 \\
			In-Class Participation & 15 \\
			Homework Assignments & 35 \\
			Final Reflection Paper & 10 \\
			Final Presentation & 35 \\ \midrule
			& 100
		\end{tabular}
	\end{center}
	\subsection{Grade Assignment}
	\begin{center}
		\begin{tabular} {r@{~--~}l l @{\hspace{0.5in}}r@{~--~}l l}
			\multicolumn{3}{c}{} & 77 & 79 &  C+ \\
		 	93 & 100 & A         & 73 & 76 &  C  \\
		 	90 & 92  & A-        & 70 & 72 &  C- \\
		 	87 & 89  & B+        & 67 & 69 &  D+ \\
			83 & 86  & B         & 63 & 66 &  D  \\
		 	80 & 82  & B-        &  0 & 62 &  F  \\
	\end{tabular}
\end{center}
\end{multicols}

\subsection{Graded Items}
\begin{description}
	\item[Attendance:] For full credit, you must show up on time.
	\item[In-Class Participation:]
		Most days will incorporate some form of discussion or activity
		in which you are expected to participate. Absences will
		negatively impact your participation grade. Of course, I know
		some days are better than others, so I will consider your
		participation over a period of at least two weeks in assigning a
		grade.
	\item[Homework Assignments:] Due to the breadth of content covered in
		this course, the style and requirements of each homework will
		vary. There will be at least 12 equally-weighted assignments
		(roughly one each week) to complete.
	\item[Final Reflection Paper:] You will be required to turn in a final
		reflection paper concerning your progress in college so far. How
		do you believe you did? What might you need to change in future
		semesters? What will you keep doing? Specific details will be
		shared towards the end of the semester.
	\item[Final Presentation:] You will identify something that you are
		curious about early in the semester and will need to research
		this topic. You will present a 5--10 minute presentation on this
		topic to the class. Additional details will be shared later in
		the semester.
\end{description}

\subsection{Assignment Submissions}
Most assignments will be turned in electronically on Brightspace. Please see
specific assignment instructions for details. Work submitted after the due date
and time will be penalized 20\,\%. For every additional 24 hours past the due
date, an additional 20\,\% will be deducted.

\subsection{Grading Errors}
It is \emph{your} responsibility to bring grading errors to my attention. You
have 72 hours from the return of the graded item to submit it for a re-grade.
\emph{The entire item will be re-graded}. You have until 4:30~p.m.\ on the last
day of the final presentations to bring clerical errors (e.g., I typed a grade
incorrectly into Brightspace) to my attention.

\section{University Policies}
As a student of Commonwealth University, you expected to know and abide by the
policies and rules set forth in the
\href{https://www.commonwealthu.edu/student-handbook/code-of-conduct}{Student
Code of Conduct} and the
\href{https://www.bloomu.edu/about/administration-and-governance/policies}{Bloomsburg
	University Policies, Procedures, and Guidelines}. Additional
	clarifications specific to this course are provided as follows.

\subsection{Attendance}
Physical presence is only part of going to class. Participation in class is
expected. You will be called upon at random and are advised to be prepared.
The best learning occurs in an active environment where everyone participates.

If you are unable to complete an assignment by its scheduled time or miss class
due to some unavoidable circumstance, you are obligated to provide a documented
valid excuse to your instructor. \emph{Failure to provide a valid excuse on
	these terms will result in a score of zero for the graded item.}

	\begin{center}
		\renewcommand\arraystretch{1.25}
\begin{tabularx}{\linewidth} {X X}
	\toprule
	\bfseries Valid Excuse & \bfseries Time Frame to Provide Documentation
	\\ \midrule
	Personal illness or accident, illness or accident of a dependent child,
	or death or critical illness of an immediate family member &
	48\,hr after the missed graded event \\
	Participation in a university-sponsored activity &
	48\,hr in advance of the missed graded event \\
	Military duty & As soon as possible \\
	Religious observance & The second Friday of the semester \\
	Others, as merited on a case-by-case basis & 48\,hr after the
	missed graded event \\
	\bottomrule
\end{tabularx}
\end{center}

Immediate technical issues, such as Brightspace being inaccessible, can be submitted to
\url{https://helpdesk.commonwealthu.edu}. Please forward your correspondance
with IT to my email address to alert me that you are taking steps to resolve the issue.

\subsection{Academic Dishonesty}
\emph{Academic dishonesty is not tolerated.} 
All work must be your own. Artificial intelligence (AI) is not to be used for
any work in this course. Work generated by AI and turned in for a grade will
result in a failing grade for the \emph{course}.

\begin{mdframed}
	\centering\bfseries The minimum penalty for academic dishonesty is a
	course assignment of ``F'' for \emph{all} students involved.
\end{mdframed}

\subsection{Technology in the Classroom}
Some assignments and units may necessitate a laptop computer. In all other
instances, if you are not using a tablet or laptop to take notes, you are
expected to pay attention and keep phones, tablets, and laptops away. Please be
respectful of others' learning by not using distracting software or websites
during class.


\vfill

\begin{mdframed}
	\noindent
	The materials contained in this syllabus and the
	Brightspace webpage for this course are intended only for those registered for
	the above course/semester. The materials cannot be used without the
	expressed written consent of Dr.\ McCurry.

	\bigskip

	\noindent
	The source code for some material has been licensed under the CC
	BY-NC-SA 4.0 license to provide you with an opportunity to view the
	original source materials and contribute to the content. If you would
	like to view the material and suggest improvements, please visit
	\url{https://github.com/danian95/FYS100}.
\end{mdframed}

\end{document}
